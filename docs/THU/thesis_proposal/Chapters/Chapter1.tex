% Chapter Template

\chapter{Introduction} % Main chapter title

\label{Chapter1} % Change X to a consecutive number; for referencing this chapter elsewhere, use \ref{ChapterX}

In "four, five years", the technology would be mature to use drones in parcel delivery,
Jeff Bezos forecasted optimistically back in 2013, 
while he was announcing Amazon Prime Air,
bringing the idea of drone delivery to the public for the first time. \cite{Hamilton2019}
Across the globe many big corporations and also smaller start-ups have
joined the idea with own research initiatives.
Nevertheless, apart from a few smaller test projects in sparsely populated areas, 
drones have not yet really been able to assert themselves in this field,
contrary to the back-then predictions of Bezos.
Decisive reasons for this are of a technical nature.
In particular, autonomous navigation methods are not yet robust enough 
for reliable deployment in densely populated urban areas. \cite{loquercio2018learning}

In science as well as in this thesis proposal, 
the colloquial term "drone" is referred to as an unmanned aerial vehicle (UAV).
Originating from the military, adopted by hobbyists,
UAVs are gaining more and more acceptance in commercial applications today.
Main areas of application are 
infrastructure, transport, insurance, media and entertainment, telecommunication, agriculture, security and mining. \cite{PwC2016}
Predominantly micro aerial vehicles (MAVs) are employed,
a sub-class of UAVs around which the research of this master thesis revolves.
MAVs navigate through the air, unaffected by obstacles on the ground and largely independent of the infrastructure. 
Difficult to access areas can be reached without great effort or even danger for the pilot.
From a bird's eye view, onboard sensors can easily record extensive data with high precision.
These capabilities pay off all the more,
the more functions MAVs perform independently, without the control of human pilots.
This is especially true for delivery applications, 
where, due to limited payloads, a MAV transports significantly fewer parcels than a conventional delivery truck.

A basic function of every flight mission is navigation, i.e.,
reaching a target position thereby avoiding obstacles and coordinating with other agents.
In undisturbed airspace in rural areas, an automated approach of GNSS-based waypoint navigation may be sufficient.
In contrast, urban areas exhibit a variety of unstructured and dynamic obstacles
as well as a high density of other agents. 
To cope with the resulting uncertainty in these environments,
a high level of autonomy is required which autonomous navigation methods have not achieved yet.
Recent research on autonomous navigation is mainly based on deep learning 
in order to deeply perceive and reason the immediate environment.
State-of-the-art navigation methods achieve a high spatial understanding of the environment
by feeding convolutional neural networks (CNN) with vision or depth data.
This research aims to develop a simple navigation method that extend this spatial perception onto temporal extension 
by serially connecting a CNN with a long-short-term-memory (LSTM) network.
Based on the assumption that powers of recall are crucial for humans when navigating,
I am convinced that future autonomous navigation systems will encompass this ability.
The navigation method will be tested in simulation and real world in a simplified test scenario,
which, however, requires the MAV to remember the expansion and relative motion of obstacles while considering its own elapsed acceleration.

This thesis proposal is structured as follows.
In chapter \ref{Chapter2} relevant literature is reviewed.
First, section \ref{sec:introduction_to_MAVs} provides a generic introduction to the aircraft class of UAVs and related concepts.
Sub-classifications of the aircraft class are presented to identify the UAV class of MAVs.
Commercial applications of UAVs with agriculture and urban delivery as examples  
are reviewed to show the enormous potential of UAV technologies
as well as currently open technologic issues that hinder the breakthrough of UAV technologies.
The concept of autonomy in the context of UAV technology is presented in general.
Second, section \ref{sec:autonomous_navigation_of_MAVs} specifically deals with autonomous navigation of MAVs.
Navigation is sub-divided into sub-tasks in order 
to show the whole scope that would have to be covered by a mature, autonomous navigation system.
Reliability and qualities of autonomous navigation are introduced to
offer comparison criteria for the design and the comparison of autonomous navigation methods.
Existing navigation methods, classical and current research, is referred to
in order to identify advantages as well as technical issues that have not been solved yet.
In chapter \ref{cha:research_project}, the intended research of this master thesis is presented.
The potential contribution of powers of recall to the robustness of autonomous navigation for MAVs is discussed.
The methodology, i.e., the general design of the unmanned aerial system, the navigation system and the test scenario,
is presented.
This proposal ends with a schedule of research and thesis writing. 








%The breakthrough is closely coupled with
%achieving high robustness of autonomous flight.
%Guaranteeing safety at high speeds and high flight performance,
%is crucial to responsibely and economically apply autonomous UAV technology
%exspecially in areas with human civilization.

%\cite{loquercio2018learning}.