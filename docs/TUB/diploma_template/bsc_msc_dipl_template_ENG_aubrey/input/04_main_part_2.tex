\chapter{Experiments}
\label{maintwo}

\section{Simulation Setup}
The experiments of this thesis 
are conducted in a drone racing simulation,
which includes the drone,
the the racetrack constituting gates 
and the scenery.
The implementation of the simulation
is devided into physics modelling and image rendering
(see figure \ref{fig:simulation_setup}).
\begin{figure}[h]
    \centering
    \includegraphics[width=1.0\textwidth]{own/simulation_setup.drawio.pdf}
    \caption[
        Implementation concept of the simulation
    ]{
        Implementation concept of the simulation
    \label{fig:simulation_setup}
    }
\end{figure}

The Gazebo\footnote{
    \url{https://gazebosim.org/home}, visited on 18/08/2022
} 
simulator is deployed to model the physics with high accuracy.
This includes the drone's dynamics and
potential collisions of the drone with the gates.
The RotorS \cite{Furrer2016} plugin
actuates the drone model with the inputted motor speed commands
and outputs the drone state estimate, the motor speed estimate and the data 
from the drone's IMU unit.
%Internally, RotorS sends the ground-truth state to the Forgetful Simulator.
%The 

The Flightmare \cite{Song2020} simulator is deployed to render 
images that are almost photo-realistic.
At a user-specified frequency,
the Flightlib interface
updates the drone's pose, and therewith its onboard camera,
within the Unity\footnote{
    \url{https://unity.com/}, visited on 20/08/2022
} Engine and fetches an RGB image from the onboard camera.
The Unity Engine is built from a Unity project
that integrates the functionalities of the RPG Flightmare Unity Project\footnote{
    \url{https://github.com/uzh-rpg/flightmare_unity}, visited on 20/08/2022
}.
The Unity project of this thesis 
and entails five scenes named
spaceship interior\footnote{
    based on the "3D Free Modular Kit" from the Unity Asset Store
},
destroyed city\footnote{
    based on "Destroyed City FREE" from the Unity Asset Store
},
industrial site\footnote{
    based on "RPG/FPS Game Assets for PC/Mobile (Industrial Set v2.0)" from the Unity Asset Store
},
polygon city\footnote{
    based on "CITY package" from the Unity Asset Store
}
and desert mountain\footnote{
    based on "Free Island Collection" from the Unity Asset Store
}
(see figure \ref{fig:unity_scenes}).
Each scene has three sites (A, B, C) to locate a racetrack.
Two different gate covers, 
one with TU Berlin/DAI-Labor and one with Tsinghua University/DME logos, 
are available (see figure \ref{fig:unity_gates}).
\begin{figure}[h]
    \centering
    \subfloat[
        Spaceship Interior
    ]{
        \label{fig:unity_scene_SI}
        \includegraphics[width=0.33\textwidth]{own/jpg/spaceship_interior.jpg}
    }
    %\hspace*{0cm}                
    \subfloat[
        Destroyed City
    ]{
        \label{fig:unity_scene_DC}
        \includegraphics[width=0.33\textwidth]{own/jpg/destroyed_city.jpg}
    }
    \par
    \subfloat[
        Industrial Site
    ]{
        \label{fig:unity_scene_IS}
        \includegraphics[width=0.33\textwidth]{own/jpg/industrial_site.jpg}
    }
    \subfloat[
        Polygon City
    ]{
        \label{fig:unity_scene_PC}
        \includegraphics[width=0.33\textwidth]{own/jpg/polygon_city.jpg}
    }
    \subfloat[
        Desert Mountain
    ]{
        \label{fig:unity_scene_DM}
        \includegraphics[width=0.33\textwidth]{own/jpg/desert_mountain.jpg}
    }
    \caption[
        Scenes implemented in simulation
    ]{
        Scenes implemented in simulation
        \label{fig:unity_scenes}
    }
\end{figure}
\begin{figure}[h]
    \centering
    \subfloat[
        DAI-Labor at TU Berlin
    ]{
        %\label{fig:unity_scene_SI}
        \includegraphics[width=0.33\textwidth]{own/jpg/tub_dai_gate.png}
    }       
    \subfloat[
        DME at Tsinghua University
    ]{
        %\label{fig:unity_scene_DC}
        \includegraphics[width=0.33\textwidth]{own/jpg/thu_dme_gate.png}
    }
    \caption[
        Gate covers implemented in simulation
    ]{
        Gate covers implemented in simulation
        \label{fig:unity_gates}
    }
\end{figure}

The Forgetful Simulator ROS node takes on three tasks.
First, it synchronizes the
drone state in the Flightmare simulator with the
ground-truth state in the Gazebo simulator.
Second, it fetches the RGB images from the Flightmare simulator
and makes them available as output.
Third, it setups the simulation
as requested by the inputted 
simulation configuration as follows.

The simulation configuration specifies the scene
(spaceship interior,
destroyed city,
industrial site,
polygon city or
desert mountain)
, the site (A, B, C),
the gate cover (TUB-DAI or THU-DME) and the racetrack configuration.
The racetrack configuration, in turn, specifies 
the type (figure-8 or gap), the generation (deterministic or randomized) 
and the direction (clockwise and counterclockwise).
The Forgetful Simulator, 
which stores the deterministic, 
counterclockwise gate poses
of the available racetrack type (see figure ...), 
processes the simulation configuration as follows.

If specified, 
the stored gate poses of the specified racetrack type are randomized in three steps.
\begin{enumerate}
    \item Shift the gate positions along the $x$-, $y$- and $z$-axis
    by a value, which is sampled, 
    independently for each gate and axis, 
    from the uniform real distribution
    over the user-specified interval 
    $\left[
        -\dist[\user]{\text{sim}}{\text{shift},\mxm}{}{},
        \dist[\user]{\text{sim}}{\text{shift},\mxm}{}{}\right]$.
    \item Scale all gate positions by the same value
    sampled from the uniform real distribution
    over the user-specified interval
    $\left[
        \dist[\user]{\text{sim}}{\text{shift},\mnm}{}{}, 
        \dist[\user]{\text{sim}}{\text{shift},\mxm}{}{}\right]$.
    \item Twist the gate yaw-orientations
    by a value, which is sampled, 
    independently for each gate, 
    from the uniform real distribution
    over the user-specified interval
    $\left[
        -\dist[\user]{\text{sim}}{\text{twist},\mxm}{}{},
        \dist[\user]{\text{sim}}{\text{twist},\mxm}{}{}\right]$.
\end{enumerate}
Then, if specified, the gate poses are redirected 
from counterclockwise to clockwise.
At last, the initial drone pose is computed to be located between
the second last and the last gate and to face towards the last gate.
The Forgetful Simulator loads the specified scene and site in 
the Flightmare Simulator
and spawns the drone model 
and the gate models (with the specified cover)
at the computed poses
in both, the Flightmare and the Gazebo, simulator.
Finally, the Forgetful Simulator
outputs the computed initial drone and gate poses,
which are required by, e.g., the expert system.




\begin{figure}[h]
    \centering
    \subfloat{
        \includegraphics[width=0.4\textwidth]{own/computeRacetrack_fig8_shift.pdf}
    }
    %\subfloat{
    %    \fbox{
    %        \begin{minipage}[c]{0.2\textwidth}
    %            Randomly shifted
    %        \end{minipage}
    %    }
    %}
    \subfloat{
        \includegraphics[width=0.4\textwidth]{own/computeRacetrack_gap_shift.pdf}
    }
    \par
    \subfloat{
        \includegraphics[width=0.45\textwidth]{own/computeRacetrack_fig8_scale.pdf}
    }
    \subfloat{ \textbf{Scale} }
    \subfloat{
        \includegraphics[width=0.45\textwidth]{own/computeRacetrack_gap_scale.pdf}
    }
    \par
    \subfloat{
        \includegraphics[width=0.45\textwidth]{own/computeRacetrack_fig8_twist.pdf}
    }
    \subfloat{ \textbf{Twist} }
    \subfloat{
        \includegraphics[width=0.45\textwidth]{own/computeRacetrack_gap_twist.pdf}
    }
    \par
    \subfloat{
        \includegraphics[width=0.45\textwidth]{own/computeRacetrack_fig8_redir.pdf}
    }
    \subfloat{ \textbf{Redirect} }
    \subfloat{
        \includegraphics[width=0.45\textwidth]{own/computeRacetrack_gap_redir.pdf}
    }
    \par
    \caption[
        Computation of the racetrack
    ]{
        Computation of the racetrack
        \label{fig:racetrack_comp}
    }
\end{figure}












\section{ANN variants}







\section{}