%#############################################################
%###################### Statement ############################
%#############################################################
\chapter*{Erkl{\"a}rung der Urheberschaft}
%this one needs to be signed for submission
Ich erkl\"are hiermit an Eides statt, 
dass ich die vorliegende Arbeit ohne Hilfe Dritter 
und ohne Benutzung anderer als der angegebenen Hilfsmittel angefertigt habe; 
die aus fremden Quellen direkt 
oder indirekt \"ubernommenen Gedanken sind als solche kenntlich gemacht. 
Die Arbeit wurde bisher in gleicher oder \"ahnlicher Form 
in keiner anderen Pr\"ufungsbeh\"orde vorgelegt 
und auch noch nicht ver\"offentlicht.


\vspace{4cm}

Ort, Datum \hfill Unterschrift

%#############################################################
%###################### Abstract  ############################
%#############################################################
\newpage
\chapter*{Abstract}
%DELETEME: An abstract is a teaser for your work. You try to convince a reader that it is worth reading your work. Normally, it makes to structure you abstract in this way: 
%\begin{itemize}
%\item one paragraph on the motivation to your topic
%\item one paragraph on what approach you have chosen
%\item and one paragraph on your results which may be presented in comparison to other approaches that try to solve the same or a similar problem.
%\end{itemize}
%Abstract should not exceed one page (aubrey's opinion)
%Autonomous drones in the middle of our daily lives are 
%the much-awaited dream of tech enthusiasts and visionary entrepreneurs, 
%but also the absolute horror for many who fear for their peace of mind, privacy or safety. 
%In fact, technological development is not yet advanced enough to safely deploy autonomous 
%drones in the uncontrolled environments of everyday life. 
%However, advances in machine learning, which can endow machines with 
%human-like capabilities, offer hope. 
Autonomous navigation methods for drones already achieve 
remarkable results using CNNs with their keen spatial comprehension 
of visual sensor data. 
Humans also rely primarily on their visual perception and spatial comprehension to move through their surroundings. 
However, unlike CNNs, their spatial comprehension 
is not limited to what is currently in their field of view, 
but extends to their memory of what they have already seen. 
They can link this memory to their often unconscious decisions 
about how to move to reach a goal. 
For example, they may incorporate their own motion history 
or their estimation/anticipation of moving objects into their decisions.
So how about trying to achieve this human-like combined, 
spatial and temporal motion competence in autonomous drone navigation?

Like much research in this area, 
this thesis draws on autonomous drone racing as a research playground. 
It takes the agile, high-performance drone racing method of Kaufmann et al. \cite{Kaufmann2018}
as a baseline and extends the feedforward CNN of the method
with a recurrent GRU to incorporate temporal comprehension 
into the navigation decision-making of the method. 
The thesis conducts experiments in simulation 
to evaluate the performance of several feedforward and recurrent variants 
in the imitation learning process of the method and in drone race tests.

Admittedly, none of the variants investigated in this work 
comes close to the exceedingly good results of the baseline work, 
mainly because of much more difficult learning/testing conditions. 
Yet, the comparison between the recurrent and feedforward variants 
studied under the same conditions shows that the recurrent variants 
perform much better in the race test, especially at higher speeds. 
At the same time, the recurrent variants are more effective in the imitation learning process, 
which manifests in significantly fewer rollouts and aggregated training samples, 
but comes at the price of significantly longer training times.


%#############################################################
%###################### German Abstract ######################
%#############################################################
\newpage
\chapter*{Zusammenfassung}
%DELETEME: translate to German to Englisch or vice-versa.

%Autonome Drohnen mitten in unserem Alltag sind der lang ersehnte Traum 
%von Technikbegeisterten und vision{\"a}ren Unternehmern, 
%aber auch der absolute Horror f{\"u}r viele, die um ihren Seelenfrieden, 
%ihre Privatsph{\"a}re oder ihre Sicherheit f{\"u}rchten. 
%Tats{\"a}chlich ist die technologische Entwicklung noch nicht weit genug fortgeschritten, 
%um autonome Drohnen in den unkontrollierten Umgebungen des t{\"a}glichen Lebens sicher einzusetzen. 
%Die Fortschritte beim maschinellen Lernen, 
%das Maschinen mit menschen{\"a}hnlichen F{\"a}higkeiten ausstatten kann, 
%geben jedoch Anlass zur Hoffnung.
Autonome Navigationsmethoden f{\"u}r Drohnen erzielen bereits bemerkenswerte Ergebnisse, 
indem sie CNNs mit ihrem scharfen r{\"a}umlichen Verst{\"a}ndnis auf visuelle Sensordaten anwenden. 
Auch Menschen verlassen sich in erster Linie auf ihre visuelle Wahrnehmung 
und ihr r{\"a}umliches Vorstellungsverm{\"o}gen, um sich in einer Umgebung zu bewegen. 
Im Gegensatz zu CNNs ist ihr r{\"a}umliches Verst{\"a}ndnis jedoch nicht auf das beschr{\"a}nkt, 
was sich gerade in ihrem Blickfeld befindet, sondern erstreckt sich auch auf ihre Erinnerung 
an das, was sie bereits gesehen haben. Sie k{\"o}nnen diese Erinnerung 
mit ihren oft unbewussten aber zielf{\"u}hrenden Bewegungsentscheidungen verkn{\"u}pfen. 
So k{\"o}nnen sie beispielsweise ihre eigene Bewegungsgeschichte 
oder ihre Einsch{\"a}tzung/Antizipation
von sich bewegenden Objekten in ihre Entscheidungen einbeziehen. 
Wie w{\"a}re es also, diese dem Menschen {\"a}hnliche 
kombinierte r{\"a}umliche und zeitliche Bewegungskompetenz 
bei der autonomen Navigation von Drohnen zu verwenden?

Wie viele andere Arbeiten in diesem Bereich nimmt auch diese Arbeit 
autonomes Drohnenrennen als Forschungsspielfeld. 
Sie nimmt die agile, hochleistungsf{\"a}hige Methode f{\"u}r Drohnenrennen
von Kaufmann et al. \cite{Kaufmann2018} als Grundlage und erweitert das 
vorw{\"a}rtsgerichtete CNN der Methode mit einer rekurrenten GRU, 
um zeitliches Verst{\"a}ndnis in die Navigationsentscheidungen 
der Methode einzubeziehen. In dieser Arbeit werden Simulationsexperimente durchgef{\"u}hrt, 
um die Leistung mehrerer vorw{\"a}rtsgerichteter und rekurrenter Varianten im 
Imitationslernprozess der Methode und in Testrennen zu bewerten.

Zwar kommt keine der hier untersuchten Varianten an die au{\ss}erordentlich 
guten Ergebnisse der Grundlagenarbeit heran, 
was vor allem an den wesentlich schwierigeren Lern-/Testbedingungen liegt. 
Der Vergleich zwischen den unter gleichen Bedingungen untersuchten rekurrenten und 
vorw{\"a}rtsgerichteten Varianten zeigt jedoch, 
dass die rekurrenten Varianten im Renntest deutlich besser abschneiden, 
insbesondere bei h{\"o}heren Geschwindigkeiten. 
Gleichzeitig sind die rekurrenten Varianten beim Imitationslernen effektiver, 
was sich in deutlich weniger Ausrollvorg{\"a}ngen und aggregierten Trainingsbeispielen {\"a}u{\ss}ert, 
aber mit deutlich l{\"a}ngeren Trainingszeiten erkauft wird.

%#############################################################
%###################### Acknowledgements #####################
%#############################################################
%\newpage
%\chapter*{Acknowledgements}
%DELETEME: Thank you for the music, the songs I am singing
%
%Yacine Zahidi: Paris Pytorch
%Lukas Kilian: Computer Unity Administratives
%HU: R{\"u}cken freigehalten, immer unterst{\"u}tzt

