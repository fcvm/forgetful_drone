\chapter{Introduction}
%labels will help you to reference to certain images, tables, chapters, section, and so on...
\label{introduction}
%DELETEME: 
%for readability purpose, 
%it makes sense to write a short paragraph on 
%what the reader can expect in this chapter.

%DELETEME: tipp: sometimes it makes sense to 
%write the first chapter, 
%the last chapter, 
%and the abstracts at the end. 
%In this case, it might be easier to argue towards your topic



%###################################################################################
%###################### Motivation          ########################################
%###################################################################################
\section{Motivation}
%DELETEME: 
%This section is very important 
%since it argues why it is necessary 
%to take care of the problem 
%you are addressing in your work. 
%One way to do this is 
%coming from a very broad view on the problem 
%to a very detailled one. 
%This can be done by establishing 
%a chain of statements that refer to each other 
%until you reach your particular problem. 
%Doing this, you really need to take care 
%for citing every statement. 

%DELETEME: 
%Example for a chain: 
%Mobile communication gets increasingly popular in the world 
%(CITE sales on mobile communication infrastruce, 
%mobile phones, or increasing number of mobile phones contracts).
%$\rightarrow$ Especially smartphones, 
%which represent the next generation cellular phone (CITE), 
%get more and more used for communicating 
%not only with other people 
%but also for connecting to the Internet 
%for using various services (CITE). 
%$\rightarrow$ Smartphone are comprehensive cellular phones 
%that provide additional functionality 
%due to their increased connection 
%and processing capabilities (CITE). 
%$\rightarrow$ Most smartphones 
%offer an online application store 
%for adding software to the devices 
%which helps the users to customize their devices 
%according to their needs, 
%e.g. Android Market\footnote{
%    \url{http://market.android.com}, visited on 05/08/2011
%}. $\rightarrow$ One problem about installing 
%third-party software is that not all softwares 
%try to help the user; 
%$\rightarrow$ software with malicious intentions, 
%so called malicious software (malware), 
%can be a severe threat to smarpthone users. 
%Some malwares delete files (EXAMPLE + CITE or footnote with URL) 
%or even destroy devices (EXAMPLE + CITE or footnote with URL). 
%$\rightarrow$ More and more smartphone malwares 
%appeared in the last years (CITE). 
%$\rightarrow$ Signature-based approaches 
%work efficiently on known malware (CITE) 
%but face serious drawbacks regarding unknown malware. 
%$\rightarrow$ Oberheide et al.~\cite{oberheide:2008:cloudav} 
%state that virus engines need an average time of 48 days 
%until their databases get updated to be able to detect 
%a certain unknown malware. 
%$\rightarrow$ This in turn means that 
%smartphone users stay unprotected for this time 
%which can be seen as a severe threat. 
%$\rightarrow$ Therefore, approaches are needed 
%that are capable of detecting unknown malware for 
%protecting the users against such threats.

%DELETEME: 
%This example showed how one could argue 
%that alternative approaches for malware detection is required. 
%The length of the motivation depends on the topics handled 
%and can of course be longer. 
%The principle I am describing is also shown on Figure~\ref{fig:writing}
%\begin{figure}
%\centering
%\includegraphics[width=0.9\textwidth]{template/writing}
%\caption[Information Generality]{This images illustrates how generality of information could be handled in a thesis. In your motivation you should start from a very broad view on the topic. Then you should get more precise with every statement until you reach the actual problem you are addressing. You should do vice-versa in your conclusion, starting with the problem that you addressed and getting broader until you can write about the meaning of your results to the (IT-)world.\label{fig:writing}}
%\end{figure}



Advances in drone technology in recent years
have opened the door to many civilian application possibilities,
especially in the commercial sector \cite{Dunn}.
Drones have already become today's standard in 
aerial inspection services
\cite{Cohn, Equinox, Abuhasira}.
In the near future, the areas of
infrastructure,
transport,
insurance,
media and entertainment,
telecommunication,
agriculture,
safety
and mining 
are considered to be particularly promising \cite{Mazur2016a}.
In contrast, 
real breakthroughs of more visionary applications, 
such as drone delivery or drone taxis
are, if at all, only expected in the more distant future \cite{Rosen2019}.
Decisive reasons for this are partly of a technical nature.
In particular, autonomous navigation methods are not yet 
robust enough for the reliable deployment in densely populated urban areas. 
\cite{brunner2019urban}
Research in the field of autonomous navigation is therefore very relevant.

Loquercio and Scaramuzza \cite{Loquercio2018a} 
categorize existing autonomous navigation methods for drones 
into classical methods and modern machine learning based methods as follows. 
%which follow the scheme of mapping-localization-planning-tracking,
%and modern methods which
%use machine learning to 
%to learn more or less end-to-end navigation policies.
%They further subdivide the machine learning approaches
%into imitation learning and reinforcement learning.

Classical autonomous navigation methods 
follow the scheme of mapping-localization-planning-tracking.
Simple approaches may localize the drone
in a given 2D map based on the data from the drone's GNSS sensor
and track preprogrammed flight mission specific waypoints.
These approaches can only be safely deployed in flight environments,
which first provide a reliable GNSS signal 
(this is often not the case, e.g., for urban areas, mountains, indoor sites and caves)
and second are free of obstacles and other agents
as functions to cope with them are absent.
More sophisticated classical approaches integrate 
simultaneous localization and mapping (SLAM) algorithms
in order to generate a map of the flight environment
and simultaneously localize the drone in this map
based on the drone's visual or range sensors. \cite{MurArtal2015}
Path planning algorithms (e.g., \cite{Bircher2016}, \cite{Cieslewski2017})
are applied to generate collision-free trajectories through the generated maps.
These trajectories are tracked while estimating the drone state
based on data from the drone's inertial measurement unit (IMU)
and the output of SLAM algorithms 
(e.g., \cite{Lin2017}, \cite{Scaramuzza2014}, \cite{Sa2018}, \cite{Loianno2017}).
Because SLAM algorithms input data from the drone's visual or range sensors,
the related autonomous navigation methods are not restricted
to flight environments with GNSS signal.
Furthermore, obstacles may be mapped and can therewith be avoided.
The main disadvantage of SLAM methods is that 
generating the map of the environment coerces global consistency,
which increases computational complexity especially for dynamic environments.
Due to the fact that drones are limited in computational power,
SLAM based methods reach their limits in dynamic environments.
Moreover, SLAM is sensitive to visual aliasing occuring at high speeds.


More recent research has publicized 
various autonomous navigation methods based on machine learning
to learn control policies that map raw vision or range sensor data 
to complex flight control commands, thereby covering
more or less of the mapping-localization-planning-tracking scope.
These methods are further subdivided 
into reinforcement learning (RL) and imitation learning (IL) problems.
The foremost benefit of deep RL methods 
is that policies are not affected by 
the control shift problem that occurs with IL methods,
because the policies are learned with "trial-and-error" \cite{Sadeghi2016} 
from direct environmental interaction without the need for expert intervention.
However, this way of learning requires an 
enormously high degree of sample complexity 
to achieve generalizing policies. \cite{Zhu2017}
For drones, limited flight endurance 
makes the learning process inefficient. \cite{Sadeghi2016}
Moreover, collisions are uncontrolled 
and thus likely to be critical for the system 
health and the safety of the environment. \cite{Sadeghi2016}
This may be circumvented 
by transferring part or all of the learning process to simulation 
while still testing the learned policies in the real world. 
For example, Sadeghi and Levine \cite{Sadeghi2016} propose a vision-based policy 
for collision avoidance in real world indoor flight 
which is exclusively trained with simulated training data.
%From raw monocular RGB images, a deep convolutional neural network directly outputs velocity motor commands.
A variety of highly randomized environments, textures and lighting, 
produces control policies that can generalize to the real world
as they proved in real-world indoor tests.
However, the drone flies with extremely low agility.
State-of-the-art simulation environments are capable of 
comprehensively modelling the dynamics of a drone. \cite{Meyer2012}
Nevertheless, complex aerodynamic effects such as rotor drag, 
which become important with the flight 
close to structures, cannot yet be modelled realisticly. \cite{Faessler2018}
Furthermore, learned policies are affected by a domain shift 
between simulation and the real world,
even in the presence of photorealistic simulators 
(e.g., AirSim \cite{Shah2017}, CARLA \cite{Dosovitskiy2017}).

Autonomous navigation methods leveraging imitation learning are,
compared to reinforcement learning, 
easier to implement and have a significantly lower sample complexity. 
This means that smaller amounts of training data 
are necessary to generalize learned control policies to test scenarios.
However, the collection of training data 
as well as the evaluation of learned control policies
may be yet inefficient and dangerous, especially in uncontrolled environments.
Therefore, alternative ways to collect training data exists.
Loquercio at al. \cite{Loquercio2018} trained a deep neural network 
with real world data that has been safely gathered by cars and 
bicycles driving through urban areas.
They achieved a control policy 
that also could safely navigate the MAV 
at high altitude in urban areas and also through indoor environments.
Similarly, Giusti et al. \cite{Giusti2016}
as well as Smolyanskiy et al. \cite{Smolyanskiy2017} 
used image data collected by human hikers.
The derived control policies could safely navigate MAVs through forest trails.
Alternatively, the learning process can be shifted to simulation,
i.e., train the network on simulated data.
Many navigation methods are restricted to planar motion 
not exploiting the agile dynamics of drones.
Ross et al. \cite{Ross2013}
developed a navigation system that inputs images from a monocular camera and
reactively controls the MAV in planar motion in order to avoid trees.
The policy, is trained to mimic yaw control of human pilot experts.
They could demonstrate collision-free flight with velocities up to 1.5 $\frac{\text m}{\text s}$
in a controlled indoor environment as well as in outdoor forest environments.
Giusti et al. \cite{Giusti2016}  used a deep convolutional neural network to map images
to the direction of forest trails relative to the MAV.
They could achieve a classification accuracy that is comparable to human decision.
However, the flight control of the MAV was restricted to planar motion (i.e., yaw and speed control)
The convolutional neural network of Loquercio et al. \cite{Loquercio2018}
outputs a steering angle and a collision probability.
The policy can follow roadways and simultaneously avoid obstacles
but is also limited to planar control, i.e., forward velocity and yaw control.


The controlled environment of autonomous drone racing 
represents a convenient ground
for basic research on autonomous navigation of drones
because it clearly defines high-level goals, 
i.e., to complete the racetrack,
as well as intermediate goals, i.e., to fly through the next race gate.
Moreover, the attainment of these goals can be intuitively assessed,
for example, by considering the racetrack completion vs. the 
maximum speed.
The 





The performance and robustness of different methods 
can be assessed intuitively. 
by means of 
Moreover, several metrics to measure performance are provided,
e.g., 

high-level goals
(i.e., completing the racetrack as fast/robust as possible)
are defined and their 
attainment can be measured (e.g., racetrack completions, flight time).
Moreover, the racing environment is controlled
and obsticles are known.

%Advanced methods for autonomous navigation of MAVs
%integrate feedforward, 
%deep convolutional neural networks (CNN)
%that, with a high spatial comprehension of the MAV's immediate surrounding, 
%map the current color or depth image to prediction or action.


%Consequently, 
%whether the state of the art in autonomous navigation of UAVs
%is sufficiently robust depends on the flight environment and the flight mission.
%Again, drone delivery provides an explanatory example here.
%Projects have been already 
%realized in rural areas where the airspace is mainly undisturbed [].
%Here, navigating through waypoints only relying on GNSS 
%without any implementation of obstacle avoidance and agent-coordination may be sufficient. 
%In contrast, urban areas are full of unstructered obstacles and other participating agents 
%which result in a high uncertainty that cannot be faced with in-advance planning. 
%Only a high level of autonomy in navigation
%could robustly cope with the challenges of this environment.
%This level has not yet been achieved.


%Research on autonomous MAV navigation mainly relies on deep learning 
%which allows to comprehend the immediate environment based on the perception of
%onboard sensors. \cite{loquercio2018learning}
%State-of-the-art navigation methods 
%achieve a high spatial understanding of the environment
%by feeding convolutional neural networks with RGB or depth data.
%This research aims to develop a simple navigation method 
%that extend this spatial perception onto temporal extension 
%by serially connecting a CNN with a long-short-term-memory (LSTM) network.
%Based on my assumption that powers of recall are crucial for humans when navigating,
%I am convinced that future autonomous navigation systems will also encompass this ability.
%The navigation method will be tested in simulation and real world in a simplified test scenario,
%which, however, requires the MAV to remember the expansion and relative motion of obstacles while considering its own elapsed acceleration.



%###################################################################################
%###################### Approach and Goals  ########################################
%###################################################################################
\section{Approach and Goals}
%DELETEME: 
%In this section, 
%you should cleary describe 
%your approach 
%that you are following 
%in order to solve 
%the underlaying problem 
%of your thesis. 
%Additionally, you should clearly state 
%the goals of your work. 
%This will not only help you supervizor 
%to understand what you are doing, 
%it will also help you to be sure 
%on which topic you should evaluate.



This thesis aims to 
investigate the question
whether temporal comprehension 
induced by a recurrent neural network
can be beneficial for the
autonomous navigation in drone racing.
To do this, the thesis uses
the vision-based autonomous navigation method
proposed by Kaufmann et al. \cite{Kaufmann2018} as a baseline.
In the tests conducted by the authors, the baseline method
stood out from the compared methods
with high reliability and agility 
along dynamic flight curves through the racetrack.
The baseline method has a hybrid structure:
an artifical neural network 
inferring navigation decisions 
from the images of the drone's onboard camera
and a planning-control stack 
translating these decisions into flight movement.
The baseline ANN is a serial connection
of a convolutional neural network (CNN) 
and fully-connected (FC) layers.
The CNN extracts visual features from imagery,
thereby enabling the baseline method
to spatially comprehend what is in the field of view of 
the drone's onboard camera.

This thesis extends the baseline ANN
with a multi-layer gated recurrent unit (GRU).
Theoretically, this
recurrent neural network variant 
enables the autonomous navigation method
to remember and to establish temporal connections.
Specifically, this means that navigation decisions 
would be made based on both current and past sensor data.
Considering that, first, 
navigation can be seen as 
sequentially making decisions 
regarding how to move through space, and second, 
the thereby resulting trajectories 
are 4D objects in space and time, 
the CNN-GRU approach of this thesis, 
by incorporating both spatial and temporal understanding, 
is expected to improve the racing performance.
This hypothesis is investigated 
in simulated drone races. 
Different variants of ANNs 
are tested for their robustness 
at different maximum speeds 
and for their ability to cope 
with intermediate target loss.
The latter refers to the event
that the next race gate intermediately leaves the FOV of the drone's camera
(e.g., due to a sharp turn between two race gates)
and the drone can only make meaningful navigation decisions
based on the memory induced by the GRU.







%According to the paper, the baseline has a success rate of 100 \% on the simulated racetrack when 
%the maximum speed is not greater than 9 m/s. For higher maximum speeds
%\{10, 11, 12\} m/s the success rate decreases to approximately \{85, 60, 35\} \%.
%Besides maximum speed, the simulated racetrack is designed in way that at any time the 
%currently targeted race gate is located in the frame of view (FOV) of the onboard camera.
%This is a requirement of the baseline because the CNN derives its navigation decisions only
%from the current image. In the event of intermediate target loss, i.e.,
%there is no race gate in the FOV and, consequently, on the image, the baseline 
%must result in undefined behaviour. 
%Intermediate target loss could for example 
%happen in the case of a steep curve between two consecutive race gates
%or in the case of an obstacle that temporally blocks the view to the currently targeted gate.
%In my thesis I plan to further develop only the first part of the hybrid approach.
%I intend to, first, replace the CNN with a R-CNN and,
%second, feeding additional features, i.e., data from the inertial measurement unit (IMU), to the network.
%The IMU data encompasses three (x, y, z) linear accelerations and three (x, y, z) angular velocities.
%I expect that thereby, 
%waypoints are not only generated on the basis of the current RGB image and IMU data, 
%but that also past sensor data is included in the navigation decision.
%This would possibly result in the following positive effects:
%\begin{itemize}
%	\item Making decisions on the basis of a series of sensor data makes the network more robust against outliers.
%			Otherwise, at high speeds, outliers may directly result in a crash.
%	\item Considering the similarity of recurrency to mathematical integration,
%			feeding IMU data to the network could have great potential since the network could be more or less able to internally estimate positions and orientations.
%	\item Due to its "memory", the network is able to generate meaningful waypoints in the case of intermediate target loss, i.e.,
%			the next gate of the race track is not depicted in the current image, but has appeared in previous images.
%	%\item Due to temporally distributed images, the network is able to take the speeds of moving gates (or obstacles) into the account of the navigation decisions.
%	\item Because trajectories are temporally extended maneuvers, a network with temporal comprehension is more able to imitate the expert system.
%			Thus, the resulting trajectory through the race track formed 
%			by the successively generated waypoints 
%			is more similar to a precomputed optimal trajectory.
%\end{itemize}
%The approach is implemented utilizing the middleware ROS \cite{ros}
%and simulated with the photo-realistic Flightmare Simulator \cite{song2020flightmare} built on Unity.
%For the implementation concept, see section \ref{sec:implConcept}.
%In simulation, the following tests should be conducted to compare my approach to the baseline.





%\paragraph{Randomized Figure-8 Drone Racing}
%Conduct test runs with increasing maximum speeds.
%For each run,
%build a randomized figure-8 drone racing track
%and test my approach and the baseline on the track.
%For each maximum speed, evaluate the percentage share of successful runs,
%the average time of racetrack completion
%and the
%closeness to the global trajectory in terms of optimality.
%The latter could be computed by recording the reference states
%that are continually pushed to the autopilot,
%taking the 4th derivative with respect to time (snap)
%and integrate the values.
%The in this way calculated cost could be used to measure closeness to
%the optimal, minimum snap global trajectory which the expert system used to generate the 
%training data.


%\paragraph{Intermediate target loss on sharp curve}
%Simulate two gates with a sharp curve in between.
%Before the drone passes the first gate,
%both successive gates must appear in the frame of view of the onboard camera.
%The curve must be so sharp, that, after traversing the first gate, not only the first but also the second gate has left the frame of view.
%The baseline, whose CNN makes navigation decisions from only the current image, will likely to fail in this scenario
%due to the absence of a goal.
%In contrast, my approach uses an R-CNN which is able to internally store information from previous images.
%In case that the R-CNN is well trained, it should be able to generate meaningful waypoints to complete the sharp curve and traverse the second gate.
%This becomes especially true, if the R-CNN is able to make use of the IMU data estimating poses.
%The percentage share of successful runs would be a convenient metric for evaluation.


%###################################################################################
%###################### Structure of the Thesis ####################################
%###################################################################################
\section{Structure of the Thesis}
%DELETEME: 
%This section does not require eloquent writing.
%It is just a presentation of what you will handle 
%in each chapter starting with Chapter~\ref{background}.

%DELETEME: Example: 
%This thesis is structured as follows. 
%In Chapter~\ref{background}, 
%we discuss essential background related to the thesis topic. 
%(SOME MORE SENTENCES). 
%Chapter~\ref{mainone} represents 
%a detailled analysis of the problem 
%that will be addressed. In particular, (SOME MORE SENTENCES). 
%In Chapter~\ref{maintwo}, 
%our solution is presented. 
%This solution covers ... (SOME MORE SENTENCES). 
%Chapter~\ref{evaluation} evaluates our solution 
%basing on our specified goals. 
%(SOME MORE SENTENCES). 
%In Chapter~\ref{conclusion}, we conclude. 
%Chapter~\ref{appendices} gives additional related information 
%on the topic of this thesis.



%DELETEME: Example: 
This thesis is structured as follows. 
Chapter~\ref{background}
provides background information on the thesis topic.
This includes the machine learning field of 
imitation learning, in particular with dataset aggregation,
and recurrent neural networks, in particular the gated recurrent unit.
Chapter~\ref{mainone} presents 
the applied autonomous drone racing navigation method,
which consists of the ANN, planning and control submodules.
Special attention is paid here to the ANN submodule,
as it is the part that enables temporal comprehension
and therewith distinguishes this thesis from the original method.
Chapter~\ref{maintwo} presents the 
experiments of the thesis.
Racing tests for the navigation method 
with different ANN submodule variants are carried out in simulation.
Thereby, the performance of the variants are measured.
Chapter~\ref{evaluation} evaluates 
the results of the racing experiments.
The focus lies on the question,
whether temporal comprehension is beneficial to the racing performance.
Chapter~\ref{conclusion} is the conclusion of this thesis. 
At last, Chapter~\ref{appendices} gives additional related information 
on the topic of this thesis.