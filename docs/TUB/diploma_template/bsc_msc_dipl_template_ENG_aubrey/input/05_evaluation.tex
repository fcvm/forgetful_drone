\chapter{Evaluation}
\label{evaluation}
%DELETEME: The evaluation chapter is one of the most important chapters of your work. 
%Here, you will prove usability/efficiency of your approach by presenting and interpreting your results. 
%You should discuss your results and interprete them, if possible. 
%Drawing conclusions on the results will be one important point that your estimators will refer to when grading your work.
This chapter consists of two sections:
the first presents results from the experiments of chapter \ref{maintwo},
the second discusses and interprets these results.


%###################################################################################
%###################### Results             ########################################
%###################################################################################
\section{Results}
\label{results}
\providecommand{\gfxwidth}{}\renewcommand{\gfxwidth}{0.8\textwidth}
%\paragraph*{Experiment 1}
This section presents the results from the four experiments conducted.
This includes, for experiment 1 and 3, the data aggregation statistics of the imitation learning process
and, for all experiments, the loss trends through the imitation/supervised learning process
and the racetrack completion shares over the maximum drone speeds from the race tests.


Table \ref{tab:e1_data} shows the data aggregation statistics 
of the imitation learning process for the variants of experiment 1.
\begin{table}[h]
    \caption[
        Data aggregation of experiment 1
    ]{
        Data aggregation of experiment 1
        \label{tab:e1_data}}        
    \centering
    \begin{tabular}{|c|r|r|r|r|r|} 
        %\hline
        \cline{2-6}
        \multicolumn{1}{c|}{}
        &F1
        &F2
        &R1
        &R2
        &R3
        \\\hline
        Learning completed
        &No
        &No
        &Yes
        &Yes
        &Yes
        \\\hline
        %
        \# Rollouts
        &297
        &208
        &114
        &136
        &178
        \\\hline
        \# Training samples
        &75,939
        &28,814
        &18,083
        &20,823
        &28,005
        \\\hline
        \# Validation samples
        &1,552
        &569
        &418
        &408
        &598
        \\\hline
    \end{tabular}
\end{table}
While all three recurrent variants (R1, R2 and R3)
completed the imitation learning process,
both feedforward variants (F1 and F2) stopped making progress and failed this task.
The learning process of F1 (or F2) was terminated 
at the 47th (72th) repetition of the rollout combination
with a maximum drone speed of 5 m/s (10 m/s) and the last margin-threshold pair of (1.0 m, 1 \%).
Consequently, F1 completed about 2/3 of the learning rollout combinations,
whereas F2 completed all but the last combination.
In the incomplete learning process, 
both feedforward variants rolled out more often
and collected more data
than all three recurrent variants in the complete learning process.
F1 performed the most rollouts with 297
and aggregated the most data with 75,939 training and 1,552 validation samples.
F2 follows by a wide margin with 208 rollouts
and 28,814 training and 569 validation samples.
In third place is F3 with 178 rollouts 
and only slightly less data with 28,005 training and 598 validation samples.
R1 (and R2) performed by far the least rollouts with 408 (418)
and aggregated by fare the least data 
with 18,083 (20,823) training and 418 (408) validation samples.






Figure \ref{fig:e1_learn} shows the 
training and the validation losses
over the epochs of the imitation learning process 
for the variants of experiment 1.
\begin{figure}
    \centering
    \subfloat[
        Feedforward variants
    ]{
        \label{fig:e1_learn_feedforward}
        \includegraphics[width=\gfxwidth]{own/results_e1_learning_feedforward.pdf}
    }
    \hspace*{0cm}
    \par
    \subfloat[
        Recurrent variants
    ]{
        \label{fig:e1_learn_recurrent}
        \includegraphics[width=\gfxwidth]{own/results_e1_learning_recurrent.pdf}
    }
    \caption[
        Training and validation losses of experiment 1
    ]{
        Training (dotted) and validation (solid) losses of experiment 1
        \label{fig:e1_learn}
    }
\end{figure}
The training or the validation loss of a variant is calculated with the SmoothL1Loss\footnote{
    \url{https://pytorch.org/docs/stable/generated/torch.nn.SmoothL1Loss.html}, visited on ??
}
function on the training or validation dataset aggregated by that variant.
The figure displays the feedforward and the recurrent variants separately,
because the feedforward variants performed much more epochs than the recurrent variants in the learning process.
While F1 and F2 went through about 3000 and 2000 epochs, respectively,
R1, R2 and R3 required about 350, 400 and 550 epochs, respectively.
At the end of the learning process,
the feedforward variants achieve roughly the same losses on their aggregated data,
which are higher than the final losses of the recurrent variants.
F1 has a final training and validation loss of 
approximately $2.6\times 10^{-4}$ and $2.2\times 10^{-3}$.
F2 has a slightly higher, final training and a slightly lower, final validation loss of approximately
$2.9\times 10^{-4}$ and $1.6\times 10^{-3}$.
R1 has the second lowest, final training and the lowest, final validation
loss of approximately $2.7\times 10^{-5}$ and $8.6\times 10^{-5}$.
R2 has the lowest final training loss and 
the highest validation loss among the recurrent variants 
of approximately $1.8\times 10^{-5}$ and $6.7\times 10^{-4}$.
R3 has a final training and validation loss of approximately $2.0\times 10^{-4}$ and $2.1\times 10^{-4}$,
which are the closest final losses to each other.
From lowest to highest,
the approximate ratios of validation to training losses are 
1.1 for R3,
3.2 for R1,
5.5 for F2,
8.5 for F1 
and 37.2 for R2.


Figure \ref{fig:e1_rcs} shows the racetrack completion shares (RCS)
over the maximum drone speeds of the race tests for the variants of experiment 1.
\begin{figure}
    \centering
    \includegraphics[width=\gfxwidth]{own/results_e1_racing.pdf}
    \caption[
        Racetrack completion shares of experiment 1
    ]{
        Racetrack completion shares of experiment 1
    \label{fig:e1_rcs}}
\end{figure}
The RCS of a variant is the share of the rollouts 
where the variant completed the racetrack
in all rollouts conducted with the same maximum drone speed in the race test of that variant.
For all variants, there is a tendency for the RCS to decrease with higher maximum speeds.
This culminates in the fact that no variant completed any racetrack 
for the highest maximum drone speed tested of 10 m/s.
At the maximum drone speed of 9 m/s, all three recurrent variants
perform better than the feedforward variants.
Over all maximum speeds, F1 has by far the lowest RCS,
which is roughly 50 \% for lower speeds from 4 to 6 m/s
and 0 \% for faster speeds from 7 to 10 m/s.
F2, R2 and R3 have roughly the same RCS over the tested maximum speeds,
which ranges from 70 to 85 \% for maximum speeds from 4 to 7 m/s
and for higher maximum speeds decreases almost linearly to 0 \% for 10 m/s.
The RCS of R1 is about the same as F2, R2 and R3
for the maximum drone speed of 4 m/s.
For maximum speeds from 5 to 9 m/s, however,
it is approximately 20 percentage points higher.





%\paragraph*{Experiment 2}
Figure \ref{fig:e2_learn} shows the 
training and the validation losses
over the epochs of the supervised learning process for the variants of experiment 2.
\begin{figure}
    \centering
    \includegraphics[width=\gfxwidth]{own/results_e2_learning.pdf}
    \caption[
        Training and validation losses of experiment 2
    ]{
        Training (dotted) and validation (solid) losses of experiment 2
    \label{fig:e2_learn}}
\end{figure}
The training or the validation loss of a variant is calculated with the SmoothL1Loss1 function 
on the final training or validation dataset of R1 in experiment 1.
All variants train for 200 epochs.
The single-layer GRU variant (R1-1x64)
achieves the lowest final training and the highest final validation loss
of approximately $5.8\times 10^{-6}$ and $1.4\times 10^{-4}$.
The 10-layer GRU variant (R1-10x64),
which has the most GRU layers in experiment 2,
achieves the highest final training and the lowest final validation loss
of approximately $2.3\times 10^{-5}$ and $6.5\times 10^{-5}$.
The variants in between (R1-2x64, R1-3x64 and R1-5x64)
achieve roughly the same final training and the same final validation loss
of approximately $1.3\times 10^{-5}$ and $1.0\times 10^{-4}$.
Consequently from lowest to highest, the approximate
ratios of validation to training losses are 2.8 for R1-10x64, 7.7 for R1-2x64, R1-3x64 and R1-5x64
and 24.1 for R1-1x64.
The validation loss of R1-10x64
converges with a higher fluctuation than the validation losses of the other variants.


Figure \ref{fig:e2_rcs} shows the RCS over the maximum drone speeds
for the variants of experiment 2.
\begin{figure}
    \centering
    \includegraphics[width=\gfxwidth]{own/results_e2_racing.pdf}
    \caption[
        Racing test performances of experiment 2
    ]{
        Racing test performances of experiment 2
    \label{fig:e2_rcs}}
\end{figure}
R1-1x64, R1-2x64, R1-3x64 and R1-5x64
have roughly the same RCS,
which ranges from 85 to 100 \%
for maximum speeds from 4 to 8 m/s,
from 65 to 85 \% for 9 m/s and 
from 5 to 25 \% for 10 m/s.
For maximum speeds from 9 to 10 m/s, however,
the RCS of R1-3x64 and R1-5x64 is approximately 10 percentage points higher than R1-1x64, R1-2x64.
Except for the maximum speed of 10 m/s,
R1-10x64 has the by far the lowest RCS,
which is about 20 percentage points lower over the tested maximum speeds.



%\paragraph*{Experiment 3}
Table \ref{tab:e3_data} shows
the data aggregation statistics of the imitation learning process for the variants of experiment 3.
\begin{table}[h]
    \caption[
        Data aggregation of experiment 3
    ]{
        Data aggregation of experiment 3
        \label{tab:e3_data}}        
    \centering
    \begin{tabular}{|c|r|r|} 
        %\hline
        \cline{2-3}
        \multicolumn{1}{c|}{}
        &E3F
        &E3R
        \\\hline
        %
        Learning completed
        &Yes
        &Yes
        \\\hline
        \# Rollouts
        &547
        &840
        \\\hline
        \# Training samples
        &25,470
        &40,216
        \\\hline
        \# Validation samples
        &None
        &None
        \\\hline
    \end{tabular}
\end{table}
Both variants, the feedforward and the recurrent variant, completed the imitation learning process.
Neither variant aggregated validation data 
because this feature had not yet been implemented when experiment 3 was conducted.
The feedforward variant (E3F) performed less rollouts with 547
and aggregated less data with 25,470 training samples
than the recurrent variant (E3R) which rolled out 840 times
and aggregated 40,216 training samples.

Figure \ref{fig:e3_learn} shows the 
training losses over the epochs of the imitation learning process for the variants of experiment 3.
\begin{figure}
    \centering
    \includegraphics[width=\gfxwidth]{own/results_e3_learning.pdf}
    \caption[
        Training losses of of experiment 3
    ]{
        Training losses of of experiment 3
    \label{fig:e3_learn}}
\end{figure}
The training loss of a variant is calculated with the SmoothL1Loss function
on the training dataset aggregated by that variant.
E3F trained for less epochs with approximately 3000
and achieved a lower final training loss of $1.3\times 10^{-5}$
than E3R with approximately 4500 and $3.2\times 10^{-5}$.
In contrast to experiment 1 (see figure \ref{fig:e1_learn}),
the variants of experiment 3
continued to train
after the imitation learning process was completed 
until the trainining loss converged.
This can be recognized by the smoother curves at the end of the loss trends.
E3R makes a larger loss drop there
because the value of the dropout probability for a single application
had to be corrected to achieve a resultant dropout probability of 50 \% for both variants.
The training loss trend of E3F fluctuates stronger than E3R
during the imitation learning process.


Figure \ref{fig:e3_learn} shows the 
RCS over the maximum drone speeds
of the race tests for the variants of experiment 3.
\begin{figure}
    \centering
    \includegraphics[width=\gfxwidth]{own/results_e3_racing.pdf}
    \caption[
        Racetrack completion shares of experiment 3
    ]{
        Racetrack completion shares
        for simulation environments
        seen (dotted) and unseen (solid) during learning
        of experiment 3
    \label{fig:e3_racing}}
\end{figure}
Thereby, the figure 
distinguishes between simulation environments seen and unseen in the imitation learning process.
In unseen environments, both variants have a RCS of about 0 \% over all maximum speeds.
In seen environments,
the E3R has a significantly higher RCS than E3F over the tested maximum speeds.
The RCS of E3F decrease almost linearly from 50 \% for 4 m/s
to 0 \% for 10 m/s.
The RCS of E3R ranges from 63 to 73 \% for maximum speeds from 4 to 7 m/s
and from there decrease almost linearly to 5\% at 10 m/s.





%\paragraph*{Experiment 4}
Figure \ref{fig:e4_learn} shows the 
training losses over the epochs of the supervised learning process for the variants of experiment 4.
\begin{figure}
    \centering
    \subfloat[
        360x240 RGB images
    ]{
        \label{fig:e4_learn_feedforward_360x240}
        \includegraphics[width=\gfxwidth]{own/results_e4_learning_360x240.pdf}
    }
    \hspace*{0cm}
    \par
    \subfloat[
        240x160 RGB images
    ]{
        \label{fig:e4_learn_recurrent_240x160}
        \includegraphics[width=\gfxwidth]{own/results_e4_learning_240x160.pdf}
    }
    \caption[
        Training losses of experiment 4
    ]{
        Training losses of experiment 4
        \label{fig:e4_learn}
    }
\end{figure}
%\begin{figure}
%    \centering
%    \includegraphics[width=\gfxwidth]{own/results_e4_learning.pdf}
%    \caption[
%        Training losses of experiment 4
%    ]{
%        Training losses of experiment 4
%    \label{fig:e4_learn}}
%\end{figure}
The training loss of a variant is calculated 
with the SmoothL1Loss function
on the final training dataset of E3R in experiment 3.
All variants trained close to convergence.
Drops in the loss trends trace back to corrections
of the dropout probability for a single application
to achieve a resultant dropout probability of 50 \% for all variants.
The variants with the RGB image size of 360x240 
have lower training losses than their counterparts 
with the RGB image size of 240x160.
For both RGB image sizes, the longer the input sequence length 
of the training samples, the lower the training loss.
The loss trend of E3R-2*360x240 fluctuates stronger than for the other variants.




Figure \ref{fig:e4_racing} shows the
RCS over the maximum drone speeds
of the race tests of experiment 4.
\begin{figure}
    \centering
    \subfloat[
        RGB image size of 360x240
    ]{
        \label{fig:e4_racing_360x240}
        \includegraphics[width=\gfxwidth]{own/results_e4_racing_360x240.pdf}
    }
    \hspace*{0cm}
    \par
    \subfloat[
        RGB image size of 240x160
    ]{
        \label{fig:e4_racing_240x160}
        \includegraphics[width=\gfxwidth]{own/results_e4_racing_240x160.pdf}
    }
    \caption[
        Racetrack completion shares of experiment 4
    ]{
        Racetrack completion shares of experiment 4
        \label{fig:e4_racing}
    }
\end{figure}
%\begin{figure}
%    \centering
%    \includegraphics[width=\gfxwidth]{own/results_e4_racing.pdf}
%    \caption[
%        Racing test results of experiment 4
%    ]{
%        Share of completed racetracks dependent on the maximum drone speed 
%        during the racing tests of the ANN module variants of experiment 4.
%    \label{fig:e4_racing}}
%\end{figure}
The figure distinguishes between simulation environments seen and unseen in the imitation learning process.
In unseen environments,
all variants have an RCS close to 0 \%,
except for E3R-10*360x240, E3R-25*360x240, E3R-2*240x160 and E3R-10*240x160,
whose RCS is up to 25, 20, 50, 35 \% for maximum speeds up to 7, 6, 8, 7 m/s, respectively.
In seen environments,
all variants have roughly the same RCS over the tested maximum speeds,
which from about 80 to 90 \% for 4 m/s decreases
almost parabolically to 0 to 15 \% at 9 m/s.
Thereby, two variants stand out.
The RCS of the starting point variant (E3R-3*360x240) 
is slightly lower at lower maximum speeds from 4 to 7 m/s 
and a slightly higher at higher speeds from 9 to 10 m/s
and the RCS of E3R-2*360x240 is below 25 \% for all tested maximum speeds.






%###################################################################################
%###################### Discussions         ########################################
%###################################################################################
\section{Discussions}
\label{discussions}
This section discusses and interprets the experimental results presented in the previous section.
Thereby, the performance of a variant is evaluated based on 
the data aggregation statistics and loss trends of the learning process
and the maximum speed-dependent racetrack completion share achieved in the race test.
The symbol $\sim$ is used to indicate that a number is an approximate value.


Experiment 1 compares the variants by their learning ability
for generalization to the randomized figure-8 racetrack in a fixed simulation environment.
The layout of the figure-8 racetrack places three demands on navigation:
the autonomous navigation method must be able to 
fly the drone through both, left and right, turns as well as across intersections.
Intersections are especially difficult,
because multiple gates appear in the FOV of the drone's onboard camera, 
which can lead to ambiguities in the navigation decision making of the variant.

The first feedforward variant of experiment 1 (F1),
which basically is the ANN used in the baseline work,
was unable to learn these navigational qualities required by the figure-8 racetrack.
As F1 stopped making progress, the imitation learning process of F1 was terminated.
During the incomplete learning process, F1 still collected
about three times as much data as the other variants of experiment 1,
with about 76k training samples.
Nonetheless, it performed by far the worst in the race tests of experiment 1
with a RCS of only 50 \% at lower maximum speeds $\le$ 6 m/s
and no completions at higher maximum speeds.
This poor performance of F1 contrasts sharply with the results 
from the simulated experiments of the baseline work
on a static racetrack and a dynamic racetrack with sinusoidally moving gates.
On the static racetrack,
the baseline method collected only 20k training samples (i.e., 56k less than F1)
and completed the static racetrack at 100 \%
for maximum speeds $\le 9$ m/s 
and at about 85, 65 and 30 \%
for the maximum drone speeds of 10, 11 and 12 m/s, respectively.
The much worse race test performance of F1
could be attributed to the more difficult learning and testing conditions in experiment 1.
While the baseline method rolls out on the same, determinstic racetrack,
F1 faces a slightly different, randomized racetracks at each rollout during 
the imitation learning process and the race tests.
However, considering the baseline results on the dynamic racetrack, 
this cannot be the only reason for the performance drop of F1 in experiment 1.
For learning to navigate through the dynamic racetrack, 
the baseline method rolled out on randomized racetracks like F1,
whereby it collected 100k training samples (i.e., 25k more than F1).
Then at the race tests,
the method completed the dynamic racetrack
with not too large movement amplitudes
for a maximum speed of 8 m/s at 100 \%.
In contrast, F1 was not able to complete a single racetrack
at maximum speeds $\ge 7$ m/s,
even though it was only tested on static racetracks.
The fact that F1 collected only 3/4 as much data as the baseline method
cannot explain the performance difference
because first,
F1 already stalled in the imitation learning process
and second, the other variants of experiment 1 
performed significantly better than F1 
even though they only collected 18-29k training samples  (i.e., 58-47k less than F1).
The main reason for the performance difference 
is likely the different layouts of the racetrack.
The racetrack of the baseline work consists of 8 gates arranged in a circle,
which requires the autonomous navigation method of only learning to fly the drone 
through either left or right turns.
In contrast, F1, is challenged by turns of both directions and intersections of the figure-8 racetrack.
Another reason can be the fact that the
simulation of this thesis is more photo-realistic than the simulation in the baseline work.
The relatively low complexity of the Resnet8 integrated with the baseline ANN and F1 
may only be sufficient to infer from the less detailed, non-photorealistic images of the baseline work.
This argument is supported by the fact that 
the other variants of experiment 1, which all integrate the more complex Resnet14,
perform significantly better than F1.
There is another distinguishing factor in experiment 1 from the baseline experiments:
F1 inputs RGB images with the size of 240x160, whereas the baseline ANN of 300x200.
However, the results of experiment 4, which examined variants of both image sizes, 
suggest that the image size has, if at all, only a marginal effect,
which could not explain the the substantial performance drop of F1.

The second feedforward variant of experiment 1 (F2),
which differs from F1 only in integrating the more complex Resnet14 instead of the Resnet8,
also stalled in the imitation learning process but 
performes much better than F1 in the race tests.
However, F2 is not yet nearly as good as the results of the baseline work indicate.
At maximum speeds $\le 8$ m/s, F2 can complete the racetrack from about 60 to 80 \%,
whereas at maximum speeds beyond, it fails for almost 100 \%.
Considering the almost equal, final training and validation losses of F1 and F2, 
the question arises why F2 outperforms F1 in the race tests,
especially since F2 collected only 29k training samples (i.e., 47k less than F1).
The answer is that F1 stalled at about 2/3 of the learning rollout combinations,
while F2 stalled only at the very last combination.
Thus, the training and the validation dataset aggregated by F2,
although containing much less samples,
better represent the state distribution experienced at the race test rollouts.
The early phases of the learning process,
where the quantity and quality of the training data is more comparable,
show that F2 can reduce the training and the validation loss more strongly and effectively than F1.
The comparison of F1 and F2
finds that a more complex CNN architecture
is able to learn more accurately
(lower losses on comparable data, same losses on more comprehensive data)
and effectively 
(less rollouts/epochs and data)
in the imitation learning process
which results in significantly better performance in the race tests.
Further experiments in the real world
could investigate, whether a drone with limited computational power 
would be able to run the more complex network.


The first recurrent variant of experiment 1 (R1)
differs from F2 in integrating three GRU
instead of three fully-connected layers.
In contrast to F2,
F1 completed the imitation learning process,
whereby R1 aggregated 18k training samples (i.e., 62.8 \% of F2)
in 114 learning rollouts (i.e., 54.8 \% of F2).
This shows that R1 needs substantially less training samples
to learn a more accurate navigation decision making than F2.
The fact that much less rollouts are needed,
makes R1 much more feasible for real-world imitation learning than F2,
where rollouts can be costly and dangerous.
Considering that the recurrent variants trained for only three
while the feedforward variant trained for ten epochs after each learning rollout,
R1 could have further reduced its number of rollouts and aggregated samples
in the comparison with F2.
However, increasing the number of epochs for R1 makes the 
learning process much more time expensive than for F2,
because a sample of R1 
has an input sequence of 25 RGB images,
whereas a sample of F2 has a single RGB image input.
In a total of 350 epochs (i.e., 15.5 \% of F2),
R1 achieved a final training loss of $2.7\times 10^{-5}$ 
(i.e., 9.3 \% of F2)
and a final validation loss of
$8.6\times 10^{-5}$
(i.e., 5.4 \% of F2).
This results in a validation-training loss ratio
of 3.2 (i.e., 58.2 \% of F2),
while both variants have a resultant dropout probability of 50 \% during training.
The much lower losses of R1 indicate
that a navigation decision is indeed temporally related
to past visual observations from the drone's onboard camera
and that R1, as it is recurrent, 
learns to leverage these underlying temporal connections
for a better navigation decision making.
The lower validation-training loss ratio of R1 
suggests that using these temporal connections
abstracts the navigation decision making more and therewith
enhances the generalization ability of R1.
R1 was able to transfer its overall better learning performance to a better race test performance.
R1 completed the racetrack, for maximum speeds from 5 to 8 m/s, at about 20 percentage points more than F2
and, for a maximum speed of 9 m/s, at about 55 percentage points more.
The comparison of F2 and R1
shows that the autonomous navigation method
performs better in learning and testing
on the randomized figure-8 racetrack
when it has a combined spatial and temporal 
(instead of a mere spatial) comprehension on its visual perception.
Considering that navigation decisions are made at 50 Hz in the experiment
and R1 trains on samples with sequence length 25,
R1 learns to have a short-term memory of 0.5 s.
In other words, R1 learns to make navigation decisions on 0.5 s videos instead of single images like F2.
Unlike a single frame, a video encodes the drone's latest motion sequence 
relative to the racetrack and the environment.
This information can, for example, resolve the ambiguities of the intersections of the figure-8 racetrack
or make the decision making more robust against single frames that as outliers 
differ significantly from the samples of the aggregated training dataset.


The second recurrent variant of experiment 1 (R2)
differs from R1 in the additional use of the optional inputs
(i.e., time steps of RGB images, IMU data also with time steps).
R2 completed the imitation learning process with $\sim 19 \%$ more rollouts
and $\sim 15 \%$ more training samples than R1.
In $\sim 15 \%$ more epochs,
R2 achieved a $\sim 33 \%$ lower final training loss (the lowest of experiment 1)
but a $\sim 679 \%$ higher final validation loss than R1.
This results in a $\sim 1063 \%$ higher, validation-training loss ratio 
(the highest of experiment 1)
than R1, while both variants have a resultant dropout probability of $50 \%$ during training.
The lower training loss of R2 suggests that the optional inputs
are somehow related to the navigation decisions and R2 learns to use these relations.
For example, R2 could have learned an anticipation mechanism,
which weights the incoming images with their time step information,
or some kind of drone state estimation based on the incoming IMU data,
which could be useful in the navigation decision making.
However, the by far larger validation-training loss ratio 
indicates that R2 heavily overfits its aggregated training data.
R2 transfers this overfitting to the race tests,
where it performs $\sim 20$ percentage points worse than R1
at the maximum speeds of 5 to 9 m/s.
The reason for the overfitting could be that the combination of both, 
the visual features in the images from the drone's onboard camera
and the physical features in the estimates from the onboard IMU,
enormously increase the drone's state space during rollout.
As a result, R2 would require much more training data than R1 for generalization.
In order to be able to make conclusive statements about the use of the optional inputs,
more experiments must be conducted.
The margin-threshold pairs of the learning configuration could be configured more stringent,
whereby a variant would be forced to collect more data to complete the learning process.
As a result, the validation-training loss ratio of R2 could become closer to R1
and R2 could outperform R1 in the race tests.

The third recurrent variant of experiment 1 (R3)
differs from R1 in the fact that 
the Resnet14 was pretrained 
and only partly trainable in the learning process.
This was a successful attempt to speed up the time of a training epoch.
However, it made the training less effective,
whereby the imitation learning process took more rollouts and epochs,
which canceled the time advantage.
R3 completed the imitation learning process with $\sim 56 \%$ more rollouts
and $\sim 55 \%$ more training samples than R1.
In $\sim 57 \%$ more epochs,
R3 achieved a $\sim 641 \%$ higher final training loss
and a $\sim 144 \%$ higher final validation loss than R1.
This results in a $\sim 66 \%$ lower validation-training loss ratio (the lowest of experiment 1) 
than R1, 
while both variants have a resultant dropout probability of $50 \%$ during training.
R3 cannot benefit from the low ratio and transfers the significantly higher losses to the race tests,
where it performs $\sim 30$ percentage points worse than R1 at the maximum speeds of 5 to 9 m/s.
The very little validation-training loss ratio of R3 
indicates that R3 does basically not overfit.
Looking at the loss trends of R3,
one can see that the trainings with the CNN submodule trainable
drastically reduce the losses but also increase the validation-training loss ratio,
while the trainings with the CNN module not trainable
slightly reduce the the losses and also the validation-training loss ratio.
This could mean that the overfitting of the other variants of experiment 1
mainly stem from the CNN submodule.
Further experiments could be conducted 
to investigate whether variants
with a trainable CNN submodule
could reduce overfitting 
with stronger regularization of the CNN submodule 
and whether a thereby reduced overfitting would reflect in the race test performance.

At a maximum speed of 9 m/s, all three recurrent variants perform better than the feedforward variants.
It is therefore likely that the higher the speed, the more beneficial the memory capabilities become for
the navigation decision making.
One explanation could be that higher speeds require higher accelerations,
which in turn require the drone to lean more into turns.
Thereby, the drone's onboard camera could produces more outlier images.
The ability to recall past images to compensate for outlier images would thus be even more valuable.
Another explaination could be that the memory time span is fixed at 0.5 s for the recurrent variants.
The higher the speed, the further the memory expands spatially.
This could make the memory more meaningful in terms of the drone's motion sequence in the race track.
All in all, experiment 1 suggests the use of a recurrent variant for the autonomous navigation method.


Experiment 2 varies R1 (from experiment 1)
with respect to the number of GRU layers 
in order to investigate
the impact of the GRU submodule's depth on performance.
Due to the high time expenditure of the imitation learning process,
the variants of experiment 2 
only train and validate with supervised learning 
on the final dataset aggregated by R1 in experiment 1.
All variants trained for 200 epochs,
whereby the training losses did not fully converge
while the validation losses almost converged.
Thereby, the number of GRU layers had only a negligible influence on the epoch time,
which is likely dominated by the loading of the sequential training data from disk.
The fact that the ten-layer GRU variant (R1-10x64)
has a $\sim 77 \%$ higher training
but a $\sim 35 \%$ lower validation loss 
than the other multi-layer GRU variants (R1-2x64, R1-3x64 and R1-5x64)
shows that the formula used to calculate the single application dropout probability
for a resultant dropout probability of $50 \%$ is only an approximation.
The single-layer GRU variant (R1-1x64) falls out of this comparison as it has no dropout.
At the race tests, R1-1x64, R1-2x64, R1-3x64 and R1-5x64 performed equally well.
They completed the racetrack at $\sim 85 - 100 \%$ for maximum speeds of 4 to 8 m/s,
at $\sim 65 - 85 \%$ for 9 m/s 
and at $\sim 5 - 25 \%$ for 10 m/s.
R1-3x64 and R1-5x64 exhibit a slightly better performance at higher maximum speeds of 9 and 10 m/s.
R1-10x64 performed significantly worse, 
with $\sim 20$ percentage points less over the tested maximum speeds.
Compared to the equally configured but less trained out R1 in experiment 1,
R1-3x64 achieved a $\sim 41 \%$ lower final training
and a $\sim 16 \%$ higher final validation loss
and performed slightly better over all tested maximum speeds.
For the comparisons of R1-10x64 vs. R1-1x64, R1-2x64, R1-3x64 and R1-5x64
as well as R1 vs. R1-3x64,
the worse performing variant has a higher final training loss and a lower final validation loss.
The training loss, thus, seems to be more important for the race test performance than the validation loss.
One reason for this could be the way data is aggregated in the imitation learning process.
The training and the validation dataset are basically a collection of mistakes made in the learning rollouts.
The fully trained variant likely avoids a lot of drone states, 
for which the validation dataset provides a correct navigation decision,
in the first place.
An important insight from experiment 2 is that 
the further training of a variant, after it completed the imitation learning process,
can enhance the race test performance.
While experiment 2 cannot make clear statements,
whether more GRU layers are generally more capable in the autonomous navigation method,
it suggest the use of R1-3x64 and R1-5x64 over R1-1x64 and R1-2x64.
Further experiments could investigate
whether R1-10x64 in the imitation learning process
could leverage its greater depth on its self-aggregated data
to outperform the other variants.



Experiment 3 focuses on the learning ability of a variant
for generalization to the randomized gap racetrack in simulation environments
seen an unseen in the imitation learning process.
The layout of the gap racetrack places two demands on the autonomous navigation method,
i.e., to fly through left or right turns and through the gap.
The gap is extremely challenging because when flying through it,
there is intermediately no gate in the FOV of the drone's onboard camera.
A feedforward ANN module could only learn to 
pass the gap by using visual information in seen environments.
A recurrent ANN module could also learn to pass the gap in unseen environments
by recalling past sensor data from its memory.
Before the drone enters the gap, both gates surrounding the gap
appear in the FOV of the drone's onboard camera.
As soon as the drone enters the gap and the FOV becomes gate-free,
the ANN module could implicitely localize the next gate relative to the drone
from its memory in order to make meaningful navigation decisions towards that gate.
Even if the memory time span is too short to cover the entire time span in which the FOV is gate-free,
after initializing the left or right turn in the covered time span,
the drone could have the memory-based awareness that it is flying a left or right turn
and could maintain the turn for the uncovered time span
until the next gate reappears in the FOV.
Experiment 3 studies a feedforward variant (E3F) and a recurrent variant (E3R).
E3F completed the imitation learning process
with $\sim 35 \%$ less rollouts and $\sim 37 \%$ less aggregated training samples than E3R.
Moreover, E3F
trained for $\sim 33 \%$ less epochs and  
achieved a $\sim 59 \%$ lower final training loss than E3R.
The better learning performance of E3F compared to E3R
is in contradiction with the results of experiment 1,
which could be explained by the fact
that the FC submodule of E3F has 
$\sim 818 \%$ more trainable parameters than the GRU submodule E3R.
In experiment 3, unlike experiment 1 conducted later,
no attention was paid to ensure
that the FC and GRU submodules of the variants
were approximately equal in the number of trainable parameters.
Despite its worse learning performance,
E3R outperforms E3F in the race tests in seen environments
by $\sim 5 - 40 \%$ percentage points over the tested maximum speeds.
In unseen environments,
both variants failed the race tests.
The performance difference in seen and unseen environments
suggests that both variants learned to use visual information in the environment to pass the gap.
The yet better race test performance in the seen environments of E3R compared to E3F
is consistent with the results and their interpretations of experiment 1.
Further experiments could be conducted,
where the variants learn in 
either a single monochrome environment
or a larger number of different simulation environments.
This could prevent the variants from learning 
visual features in the environment
and could force them to use their memory to pass a gap.




Experiment 4 varies E3R (from experiment 3)
with respect to the input sequence length of the training samples
and the input RGB image size at training and rollout
in order to investigate their effects on performance.
The variants of experiment 4 
share the same learning capabilities, because their ANN modules are configured the same,
but are provided with different ressources to learn from.
Due to the high time expenditure of the imitation learning process, 
the variants of experiment 4 trained with supervised learning.
Each variant trained on its own training dataset,
that had been rebuilt from the raw records of E3R 
with the corresponding sequence length and image size of that variant.
All variants trained until near convergence.
For variants of the same image size,
the longer the sequence length,
the lower the training loss of the variant.
This substantiates the interpretation from experiment 1
that navigation decisions are temporally connected to past sensor data
and that these connections are learnable by recurrent variants.
For the range of investigated sequence lengths,
the longer the memory time span,
the more the variant can learn.
For variants of the same sequence length,
the larger the image size, the lower the training loss.
This shows that a more detailed, visual observation 
is a stronger ressource to learn the navigation decision making.
The race test performances of the variants are roughly the same,
whereby lower training losses do not reflect in higher performances.
The variants with the smaller image size perform 
slightly better in unseen environments than the 
variants with the bigger image size.
A reason for this could be that, the ratio
of spatial to temporal information in the training samples
decreases with the image size,
which favors the learning of temporal before spatial features,
which are necessary to pass the gap in unseen environments as mentioned above.
Further experiments
could find whether a variant
trained with imitation learning
can leverage its lower training loss
to achieve a better race performance.


































