\chapter{Motivation}

{\color{cyan}Im Bereich ... (Warum muss es eine neue L"osung/ einen neuen Ansatz geben)}




The advances in drone technology 
over the past years have been just as fantastic as 
associated application possibilities in the 
public, private and voluntary sector of today's and future economies.
Especially the branches 
infrastructure, transport, insurance, media and entertainment, 
telecommunication, agriculture, security and mining 
offer great potential for the commercial use of drones. \cite{PwC2016}
Drones exhibit substantial advantages over ground vehicles
as they are unaffected by many obstacles and largely independent of infrastructure.
Destinations can be reached by the shortest route, waiting times can be avoided
and the effort and risk of getting to places that are difficult to access can be reduced.
In addition, from the privileged perspective of a bird, onboard sensors 
are able to record extensive data 
of high quality as well as increased speed 
due to the fast aerial maneuverability of drones.
On the downside,
drones as smaller aircraft vehicles can move much less weight
which, among other things, limits mission payload and battery capacity,
which in turn restricts flight range and duration.
In order for the use of drones to pay off economically, efficiency must therefore be maximized.
One key to increase efficiency is to shift functions from a human operator to the drone itself,
i.e., increase the drone's degree of autonomy.
This becomes particularly clear in the example of delivery applications.
Because drones can load significantly fewer packages and
have to recharge or refuel more often than a conventional delivery truck,
drone delivery systems would only really pay off if
most decisions are made autonomously by the individual delivery drone
or even by the collective of the drone swarm.
Consequently, expensive human labor could be reduced
and room for mathematical optimization could be created.

%https://www.equinoxsdrones.com/blog/visual-drone-inspection-across-different-industries
%https://percepto.co/in-2022-percepto-is-bringing-visual-inspection-automation-to-all/
%https://www.munichre.com/topics-online/en/mobility-and-transport/drone-use-taking-flight-on-small-farms.html
While in the commercial context,
already today drones are proving their economic and safety-related value
in aerial inspection services of
more controlled and undisturbed 
environments of "large industrial sites" such as
agriculture,
construction,
infrastructure,
utilities,
and mining,
they have not yet have not yet really 
been able to assert themselves for 
commercial applications in the open-world environment
of our daily lives.
This is particularly true for drones with
autonomous functions.


drone-in-a-box %https://en.wikipedia.org/wiki/Drone_in_a_Box
-In heavy industry, automation and drone use is now seen as the standard in inspection practices
-Our 2021 solution, combining powerful autonomous robots with AI-powered visual data management is the nuanced and advanced solution industrial sites today can rely on to reduce risk, costs and environmental impact.

Apart from legal restrictions and lack of acceptance in the population [Quelle], 
decisive reasons for this are of a technical nature.
For example, autonomous navigation methods are not yet robust enough
for reliable deployment in densely populated urban areas.\cite{loquercio2018learning}
The master's thesis is intended to make a contribution here.


-----------BIS HIER



In common parlance, the term "drone" is often used to refer 
to the aircraft class of unmanned aerial vehicles (UAVs),
which can be further devided into the sub-classes of


This proposal deals with the sub-class of micro aerial vehicles (MAVs)
which represent ...
.
Classification...
Current research revolves around... batteries, lalala
and more autonomy, especially in navigation.





As a basic functionality of MAVs, navigation
comprises the main task of achieving a desired pose or position
while performing necessary sub-tasks at the same time.
The necessity of implementing individual sub-tasks depend either on the environment,
e.g., obstacle avoidance and coordination with other agents,
or on the application,
e.g., shooting plant seeds [].
Consequently, state of the art allows the degree of autonomy 
of MAVs in some environments and applications and not in others.

When again taking drone delivery as an example,
projects are already realized in rural areas where the airspace is mainly undisturbed [].
Here, navigating through waypoints only relying on GNSS 
without any implementation of obstacle avoidance and agent-coordination may be sufficient. 

In contrast, urban areas are full of unstructered obstacles and other agents 
which result in a high uncertainty that cannot be planned in advance. 

Only a high level of autonomy in navigation which has not yet been achieved
can robustly cope with the challenges of this environment.


Recent research on autonomous MAV navigation is mainly based on deep learning 
which allows to perceive and reason the immediate environment.
State-of-the-art navigation methods achieve a high spatial understanding of the environment
by feeding convolutional neural networks (CNN) with vision or depth data.

This research aims to develop a simple navigation method 
that extend this spatial perception onto temporal extension 
by serially connecting a CNN with a long-short-term-memory (LSTM) network.

Based on the assumption that powers of recall are crucial for humans when navigating,
I am convinced that future autonomous navigation systems will also encompass this ability.
The navigation method will be tested in simulation and real world in a simplified test scenario,
which, however, requires the MAV to remember the expansion and relative motion of obstacles while considering its own elapsed acceleration.




\chapter{Objectives}
Im Rahmen ... (Was will ich ueberhaupt mit meiner Abreit erreichen? Etwas verbessern, entwickeln, vergleichen...)

 
The following objectives should be achieved within the framework of the master's thesis:



Many advanced methods for autonomous navigation of MAVs already exist.
However, they are not sophisticated enough to conquer open-world environments of high uncertainty.
In current research, deep learning techniques
that empower MAVs with perception and reasoning abilities
constitute the best approach to face the uncertainty of these environments.


State-of-the-art, vision- or depth-based navigation methods 
integrate feedforward, deep convolutional neural networks (CNNs)
that map the current color or depth image to action. [Sources]

By deploying CNNs in this way a high, spatial perception and reasoning 
of the immediate surrounding of the MAV can be achieved.
Yet, the mere comprehension of space, however good it is,
may not be sufficient to robustly deploy autonomously navigating MAVs in 
open-world environments.
Therefore, I aim to develop a navigation method that besides spatial
also includes temporal comprehension.


To my knowledge, researchers from ETH Zurich have come up with very impressing work 
with respect to autonomous MAVs.

Current paper: ...

Want to take a step back to the work of Loquercio, Kaufmann et. al. \cite{Kaufmann2018} from 20??:
They developed a vision-based method that navigates a MAV through a drone racing 
track with possibly dynamically moving race gates.
Thereby, they achieved a high reliability and agility at high speeds.

drone racing is a good test environment...
- reactive, inherent problem of high-level goal formulation in deep learning policies

Taking their work as a basis,
by adding temporal comprehension,
I expect the MAV to entfalten das folgende Potential:

- As mentioned above, the navigation method theoretically does not require a high-level goal formulation
since the reactive targeting the next gate already results in completing the race track.
Practically, in the event that at any time the next gate is not located 
in the frame of view (FOV) of the MAV's onboard camera,
the output of the deep neural network (DNN) is not defined and the lack of a high-level goal becomes evident.
This is also the case, even if the MAV has seen the next gate in the past (see figure ??).

- In addition, humans cannot estimate the velocity and direction of motion of themselves, obstacles or other agents
based on a short blink with their eyes but they need to observe over at least a short period of time.
- Not only localization but also situational reasoning is strengthened by memory,
e.g., a car driver observes a child that runs from the sidewalk through the parking cars onto the road
and has enough reaction time or even anticipation to brake.
Without any power of recall, the driver may start braking in anticipation
when the child is still on the sidewalk, but may stop braking after the child disappears behind a parking car.

- dynamic trajs

In the navigation method of this research, I want to meet the lack of high-level goal planning 
by introducing powers of recall to navigation.
To my knowledge, Kelchtermans and Tuytelaars \cite{Kelchtermans2017}
are the only ones who used a recurrent neural network for memory abilities in autonomous, 
vision-based UAV navigation. \cite{Shakeri2019}
-reason why their work is not what i want..
However, they only tested their method in simulation and did not comprehensively evaluate their results.
If the method, after passing the first gate, could remember that it has seen the second gate before,
the method could plan to navigate through the second gate based on elapsed images.




In my method, I plan to use a LSTM-CNN which is a deep convolutional neural network
serially connected to a long-short-term-memory (LSTM) neural network.

While the CNN has the ability to perceive and reason spatial structures of the immediate environment,
the LSTM is able to establish connections through time. 

In other words, the CNN is responsible to predict waypoints 
or generate trajectories based on single images,
whereas the LSTM empower the method to recall and remember 
by evaluating the temporal structure between the predictions of the CNN.

Besides the above steep curve scenario, this memory would show great benefit in situations
when the quadcopter lost track of the goals and it can recall elapsed images.
In case of the application in urban areas, for example, 
quadcopters could remember obstacles 
that were visible but have become occluded and thus, could better anticipate. 
Or after an evasive maneuver, 
the quadcopter could return to the actual path 
much faster because he memorizes its maneuver.
In that sense, the memory of the quadcopter is another form of localization in the environment, 
which is not global but namely local.
In addition, memory could enable better optimization, e.g., 
the imitation learning of optimal trajectories which
are not only spatial but also temporal objects.
However, this research, in a simplified scenario, should only prove if powers of recall 
are applicable and generally useful for the autonomous navigation of MAVs.

\chapter{Task Packages}
Based on the objectives, the following task packages are defined:

\begin{itemize}
	\item Familiarization with the subject area. 
	This includes evaluating relevant papers and 
	taking into account the components and implementations 
	that are already available. 
	The work result is a summary of relevant work.

	\item Developing an Approach ...

	\item Analysis, design and drafting of an architecture

	\item Definition of suitable test scenarios

	\item Implementation of the architecture, implementation

	\item Documentation of the architecture and the program code
	
	\item ------------------------------------------------------
	\item Shift simulation from Gazebo to the more photo-realistic Flightmare
	\item Generate training data in simulation
	\item Shift data input pipeline from tensor flow to pytorch
	\item Design and drafting of the RCNN in pytorch
	\item Train the RCNN
	\item Definition of suitable test scenarios in simulation
	\item Implementation
	\item Documentation of the architecture and the program code

\end{itemize}

	
\chapter{Time Schedule}
Die Bearbeitung dauert maximal 6 Monate. (Am besten eignet sich ein Gannt Diagramm)

\chapter{Organizational Matters}
\begin{itemize}
	\item Language of the thesis: English
	\item Text processing system: LaTeX
	\item Programming languages: C++, Python
	\item Supervisor: Dr. rer. nat. Yuan Xu
	\item Reviewers: Prof. Dr. Sahin Albayrak, Dr.- Ing. Stefan Fricke
\end{itemize}

\chapter{Annex}

\section{Steep curve scenario}

For example, a section of the racetrack that consist of two successive gates, in between a steep curve
could not successfully be navigated through by the method, even if both gates have already appeared on images.
Before the MAV navigate through the first gate, both gates are in the FOV of the camera.
After it has flown through the first gate, because the curve is too steep, the second gate is out the FOV 
and the navigation method has no goal to be achieved.

% Taking a human as example
% HUMAN IN THE DRONE RACETRACK EXAMPLE
%To solve this problem I consider what a human would do in this scenario.
%From a position before entering the gate 1, a human can see gate 1 and 2.
%After he flew through gate 1, he has no visual contact to both gates anymore.
%Yet, he can manage to navigate through gate 2 because he has the ability 
%to remember the position of gate 2 in his body frame. 


